En este capítulo se abordará como preparar el entorno para instalar y ejecutar el comando `gh owner` en la GitHub CLI. Además de explicar el desarrollo del comando y las dificultades encontradas.

\section{Entorno}
\subsection{Instalar GO}
El primer paso para utilizar el comando es asegurarse de que Go (Golang) está instalado en el sistema. Para ello, siga estos pasos:

\begin{enumerate}
  \item Descargue la última versión de Go desde la página oficial: https://golang.org/dl/.
  \item Siga las instrucciones de instalación correspondientes a su sistema operativo (Windows, macOS, Linux).
  \item Verifique la instalación abriendo una terminal y ejecutando el siguiente comando:
        \begin{verbatim}
          go version
        \end{verbatim}
\end{enumerate}

Si Go está instalado correctamente, verá una salida con la versión de Go instalada.

\subsection{Descargar el repositorio}

Dado que este proyecto no es una modificación directa de la GitHub CLI oficial, es necesario clonar el repositorio desde un fork y acceder a la rama de desarrollo. Para ello, siga los siguientes pasos:

\begin{enumerate}
  \item Abra una terminal y clone el repositorio desde el fork con el siguiente comando:
        \begin{verbatim}
          git clone git@github.com:gh-cli-for-education/cli.git
        \end{verbatim}
  \item Acceda al directorio del repositorio clonado.
        \begin{verbatim}
          cd cli
        \end{verbatim}
  \item Cambie a la rama de desarrollo específica:
        \begin{verbatim}
          git checkout gh-owner-dev
        \end{verbatim}
\end{enumerate}

\subsection{Compilar el proyecto}

Una vez dentro del repositorio clonado, puede compilar el proyecto utilizando el comando `make`. Siga estos pasos:

\begin{enumerate}
  \item Asegúrese de estar en el directorio raíz del repositorio clonado.
  \item Ejecute el comando `make` para iniciar el proceso de compilación. Este comando se encargará de construir el proyecto y generar los binarios necesarios.
  \item Después de la compilación exitosa, puede utilizar el binario generado para ejecutar el comando. En lugar de utilizar el comando `gh` estándar, utilice el binario ubicado en el directorio `bin` del proyecto:
        \begin{verbatim}
          ./bin/gh <subcomando>
        \end{verbatim}
\end{enumerate}

Con estos pasos, el entorno estará configurado correctamente y podrá empezar a utilizar el comando desarrollado.

\section{Desarrollo del comando}

