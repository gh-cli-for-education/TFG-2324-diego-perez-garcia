En este capítulo, se describirá cómo instalar y ejecutar el comando \texttt{gh owner} en la CLI de GitHub. Cabe destacar que el comando \texttt{gh owner} no ha sido añadido a la CLI oficial de GitHub, por lo que este capítulo sirve como una demostración. Posteriormente, se explicará la implementación del comando y su ejecución.

\section{Instalar \texttt{gh}}

Para utilizar el comando \texttt{gh owner}, primero es necesario instalar la GitHub CLI (\texttt{gh}). A continuación, se detallan los pasos para su instalación en diferentes sistemas operativos.

\subsection{Instalación en macOS}

\begin{enumerate}
    \item \textbf{Usando Homebrew}:
    \begin{verbatim}
    brew install gh
    \end{verbatim}
    \item \textbf{Usando MacPorts}:
    \begin{verbatim}
    sudo port install gh
    \end{verbatim}
\end{enumerate}

\subsection{Instalación en Windows}

\begin{enumerate}
    \item \textbf{Usando el instalador}:
    \begin{itemize}
        \item Descargue el \href{https://github.com/cli/cli/releases}{instalador} desde la página de lanzamientos de GitHub CLI.
        \item Ejecute el archivo \texttt{.msi} y siga las instrucciones.
    \end{itemize}
    \item \textbf{Usando Scoop}:
    \begin{verbatim}
    scoop install gh
    \end{verbatim}
    \item \textbf{Usando Chocolatey}:
    \begin{verbatim}
    choco install gh
    \end{verbatim}
\end{enumerate}

\subsection{Instalación en Linux}

\begin{enumerate}
    \item \textbf{Usando un paquete Debian/Ubuntu}:
    \begin{verbatim}
    sudo apt install gh
    \end{verbatim}
    \item \textbf{Usando un paquete Fedora}:
    \begin{verbatim}
    sudo dnf install gh
    \end{verbatim}
    \item \textbf{Usando un paquete Arch Linux}:
    \begin{verbatim}
    sudo pacman -S github-cli
    \end{verbatim}
    \item \textbf{Descargando y compilando desde el código fuente}:
    \begin{itemize}
        \item Clonar el repositorio:
        \begin{verbatim}
        git clone https://github.com/cli/cli.git
        cd cli
        \end{verbatim}
        \item Compilar y instalar:
        \begin{verbatim}
        make
        sudo make install
        \end{verbatim}
    \end{itemize}
\end{enumerate}

\section{Ejecutar \texttt{gh owner}}

Una vez instalada la GitHub CLI, se puede ejecutar el comando \texttt{gh owner}. A continuación, se describe cómo utilizarlo y las distintas opciones disponibles.

\subsection{Uso básico}

El comando \texttt{gh owner} permite gestionar el propietario predeterminado para los comandos de la CLI de GitHub.

\begin{verbatim}
gh owner [OWNER] | [flags]
\end{verbatim}

\subsection{Flags}

\begin{itemize}
    \item \texttt{-l}, \texttt{--list}: Lista las organizaciones.
    \item \texttt{-s}, \texttt{--select}: Selecciona interactivamente un propietario predeterminado.
    \item \texttt{-u}, \texttt{--unset}: Desestablece el propietario predeterminado.
\end{itemize}

\subsection{Flags heredados}

\begin{itemize}
    \item \texttt{--help}: Muestra la ayuda para el comando.
\end{itemize}

\subsection{Argumentos}

Se puede especificar un propietario como argumento o usar los flags \texttt{--list}, \texttt{--select}, o \texttt{--unset}. Cualquiera de estas opciones puede ser utilizada, pero solo una a la vez.

\section{Ejemplos de uso}

A continuación, se presentan algunos ejemplos prácticos de cómo utilizar el comando \texttt{gh owner}.

\subsection{Sin argumentos}

Este comando muestra el propietario predeterminado actual.

\begin{verbatim}
$ gh owner
\end{verbatim}

\subsection{Especificar un propietario}

Este comando establece un propietario específico como predeterminado.

\begin{verbatim}
$ gh owner [ORGANIZATION | USER]
\end{verbatim}

\subsection{Listar organizaciones}

Este comando lista todas las organizaciones disponibles.

\begin{verbatim}
$ gh owner --list
\end{verbatim}

\subsection{Selección interactiva de propietario}

Este comando permite seleccionar interactivamente un propietario predeterminado.

\begin{verbatim}
$ gh owner --select
\end{verbatim}

\subsection{Desestablecer el propietario predeterminado}

Este comando desestablece el propietario predeterminado actual.

\begin{verbatim}
$ gh owner --unset
\end{verbatim}

\section{Aplicaciones sobre otros comandos}

El comando \texttt{gh owner} se puede utilizar para establecer un propietario predeterminado, lo cual simplifica el uso de otros comandos que reciben \texttt{OWNER/REPO} como argumento. A continuación, se muestra un ejemplo práctico de cómo aplicar este concepto.

\subsection{Seleccionar un propietario predeterminado}

Primero, seleccionamos \texttt{gh-cli-for-education} como el propietario predeterminado:

\begin{verbatim}
// Seleccionamos gh-cli-for-education como OWNER
gh owner gh-cli-for-education
\end{verbatim}

\subsection{Usar comandos con \texttt{OWNER/REPO}}

Una vez que hemos establecido el propietario predeterminado, podemos usar cualquier comando de la GitHub CLI que reciba \texttt{OWNER/REPO} como argumento sin necesidad de especificar el propietario cada vez. Por ejemplo, para navegar al repositorio \texttt{cli} del propietario predeterminado, usamos:

\begin{verbatim}
// Usamos cualquier comando que reciba OWNER/REPO como argumento
gh browse --repo cli
\end{verbatim}

En este caso, el comando \texttt{gh browse} abrirá el repositorio \texttt{gh-cli-for-education/cli} en el navegador predeterminado.

Estos pasos muestran cómo el comando \texttt{gh owner} facilita la interacción con otros comandos de la GitHub CLI al gestionar el propietario predeterminado de manera eficiente.
