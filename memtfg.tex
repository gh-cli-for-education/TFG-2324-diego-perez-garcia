\documentclass[spanish,a4paper,12pt,oneside]{extreport}

\usepackage[dvips]{graphicx}
\usepackage[dvips]{epsfig}

\usepackage[T1]{fontenc}
\usepackage[utf8]{inputenc}
\usepackage[spanish]{babel}

%\usepackage[latin1]{inputenc}
%\usepackage[spanish]{babel}

\usepackage{alltt}

\usepackage[ruled,vlined,commentsnumbered,linesnumbered,inoutnumbered,titlenotnumbered,noend]{algorithm2e}
\SetKwRepeat{Do}{do}{while}

\usepackage{multirow}
\usepackage{array} 
\usepackage{amsfonts}
\usepackage{amsmath}
\usepackage{bigstrut}
\usepackage{booktabs}
\usepackage{caption}
\usepackage{chngpage}
\usepackage{float}
\usepackage{comment}
%\usepackage{enumitem,lipsum}
\usepackage{graphicx}
\usepackage{lscape}
\usepackage{microtype}
\usepackage{natbib}
\usepackage{pdflscape}
\usepackage{rotating}
\usepackage{subcaption}
\usepackage{ctable}
\usepackage{hyperref}
\usepackage{enumerate}
\usepackage{gensymb}
\usepackage{eurosym}
\usepackage{xcolor}
\usepackage{tabu}
\usepackage[export]{adjustbox}[2011/08/13]
\usepackage{lineno}
\usepackage{epigraph}
%\usepackage[sanitize=none]{glossaries}
\usepackage[toc]{glossaries}

%\\linenumbers
%\setlength\linenumbersep{5pt}
%\renewcommand\linenumberfont{\normalfont\tiny\sffamily\color{gray}}

%\makenoidxglossaries
\makeglossaries

\newglossaryentry{DSL}{
    name={DSL},
    description={
        Un lenguaje específico de dominio (DSL) es un lenguaje que está dirigido a resolver un tipo particular de problema. El uso de DSL es muy común en informática: CSS, expresiones regulares, make, ant, SQL, muchos bits de Rails, expectativas en JMock, el lenguaje graphviz, los archivo de configuración de strut, etc.
    }
}

%\newglossaryentry{GitHub Action}{
%    name={GitHub Action},
%    description={
%        GitHub Actions es una plataforma de integración continua y entrega continua (CI/CD) que permite automatizar la %canalización de tareas como la compilación, pruebas y despliegue, permitiendo crear flujos de trabajo que construyan %nuevos artefactos como publicaciones, releases, imágenes, pdfs, etc.
%    }
%}

\newglossaryentry{JIT}
{
    name=JIT,
    description={just-in-time compiler. También conocido en español como \textbf{compilación en tiempo de ejecución}. Es una forma de ejecutar el código informático que implica la compilación durante la ejecución de un programa (en tiempo de ejecución) en lugar de antes de la ejecución}
}

\newglossaryentry{AOT}
{
    name=AOT,
    description={En informática, la \textbf{compilación anticipada} es el acto de compilar un lenguaje de programación (a menudo) de alto nivel en un lenguaje (a menudo) de bajo nivel antes de la ejecución de un programa, normalmente en tiempo de compilación, para reducir la cantidad de trabajo necesario en tiempo de ejecución.
    De hecho, dado que todas las compilaciones estáticas se realizan técnicamente con antelación, esta expresión en particular se utiliza a menudo para destacar algún tipo de ventajas de rendimiento del acto de dicha precompilación. El acto de compilar Java a bytecode Java es por tanto raramente referido como AOT ya que es normalmente un requisito, no una optimización}
}

\newglossaryentry{API}
{
    name=API,
    description={Una \textbf{interfaz de programación de aplicaciones} es una manera de que dos o más programas informáticos se comuniquen entre sí. Es un tipo de interfaz de software que ofrece un servicio a otras piezas de software}
}

\newglossaryentry{power users}
{
    name=power users,
    description={Un \textbf{usuario avanzado} es un usuario de ordenadores, software y otros dispositivos electrónicos, que utiliza características avanzadas del hardware informático, sistemas operativos, programas o sitios web que no son utilizados por el usuario medio. Un usuario avanzado puede no tener amplios conocimientos técnicos de los sistemas que utiliza, pero se caracteriza más bien por la competencia o el deseo de hacer el uso más intensivo de los programas o sistemas informáticos}
}

\newglossaryentry{cache}
{
    name=caché,
    description={Una \textbf{cache} es un componente de hardware o software que almacena datos para que las futuras solicitudes de esos datos puedan ser atendidas más rápidamente; los datos almacenados en una caché pueden ser el resultado de un cálculo anterior o una copia de datos almacenados en otro lugar}
}

\newglossaryentry{CLI}
{
    name=CLI,
    description={Es un tipo de interfaz de usuario de computadora que permite a los usuarios dar instrucciones a algún programa informático o al sistema operativo por medio de una línea de texto simple}
}

\newglossaryentry{strategy pattern}
{
    name=patrón estrategia,
    description={En programación informática, el patrón de estrategia (también conocido como patrón de política) es un patrón de diseño de software de comportamiento que permite seleccionar un algoritmo en tiempo de ejecución. En lugar de implementar un único algoritmo directamente, el código recibe instrucciones en tiempo de ejecución sobre cuál de una familia de algoritmos utilizar}
}

\newglossaryentry{graceful degradation}{
    name={degradación gradual},
    description={La \textbf{tolerancia a los fallos} es la propiedad que permite a un sistema seguir funcionando correctamente en caso de fallo de uno o varios de sus componentes. A diferencia de un sistema diseñado ingenuamente, en el que incluso un pequeño fallo puede provocar una avería total}
}

\newglossaryentry{namespace}{
    name={espacio de nombre},
    description={En informática, un \textbf{espacio de nombres} es un conjunto de signos (nombres) que se utilizan para identificar y referirse a objetos de diversos tipos. Un espacio de nombres garantiza que todos los objetos de un conjunto determinado tengan nombres únicos para que puedan ser fácilmente identificados}
}

\newglossaryentry{corrutine}{
    name={corrutina},
    description={Las \textbf{corrutinas} son componentes de programas informáticos que generalizan las subrutinas para la multitarea no preferente, permitiendo suspender y reanudar la ejecución}
}

\newglossaryentry{hash}{
    name={hash},
    description={Un \textbf{hash} (que también puede llamarse un condensado o una firma) es un valor calculado a partir de un valor diferente. En la gran mayoría de los casos, este valor está representado en forma de una cadena de caracteres hexadecimales. Los hashs suelen utilizarse para verificar que un archivo no se ha visto corrompido o bien para autentificar a un usuario sin tener que almacenar su contraseña en claro}
}

\newglossaryentry{regex}{
    name={expresión regular},
    plural={expresiones regulares},
    description={Una \textbf{expresión regular} (abreviada como regex o regexp) es una secuencia de caracteres que especifica un patrón de búsqueda en un texto. Los algoritmos de búsqueda de cadenas suelen utilizar estos patrones para las operaciones de "búsqueda" o "búsqueda y sustitución" de cadenas, o para la validación de entradas}
}

\newglossaryentry{deserialization}{
    name={deserializar},
    plural={deserializa},
    description={Extraer una estructura de datos de una serie de bytes}
}

\newglossaryentry{serialization}{
    name={serializar},
    plural={serializa},
    description={Es el proceso de traducir una estructura de datos o el estado de un objeto a un formato que pueda ser almacenado (por ejemplo, en un archivo o en un búfer de datos de la memoria) o transmitido (por ejemplo, a través de una red informática)}
}

\usepackage[top=2cm, bottom=2cm, left=2cm, right=2cm]{geometry}

\newenvironment{sourcecode}
{\begin{list}{}{\setlength{\leftmargin}{1em}}\item\scriptsize\bfseries}
{\end{list}}

\newenvironment{littlesourcecode}
{\begin{list}{}{\setlength{\leftmargin}{1em}}\item\tiny\bfseries}
{\end{list}}

\newenvironment{summary}
{\par\noindent\begin{center}\textbf{Abstract}\end{center}\begin{itshape}\par\noindent}
{\end{itshape}}

\newenvironment{keywords}
{\begin{list}{}{\setlength{\leftmargin}{1em}}\item[\hskip\labelsep \bfseries Keywords:]}
{\end{list}}

\newenvironment{palabrasClave}
{\begin{list}{}{\setlength{\leftmargin}{1em}}\item[\hskip\labelsep \bfseries Palabras clave:]}
{\end{list}}

\usepackage{bera}% optional: just to have a nice mono-spaced font
\usepackage{listings}
\usepackage{xcolor}

\colorlet{punct}{red!60!black}
\definecolor{background}{HTML}{EEEEEE}
\definecolor{delim}{RGB}{20,105,176}
\colorlet{numb}{magenta!60!black}

\lstdefinelanguage{json}{
    basicstyle=\normalfont\ttfamily,
    numbers=left,
    numberstyle=\scriptsize,
    stepnumber=1,
    numbersep=8pt,
    showstringspaces=false,
    breaklines=true,
    frame=lines,
    backgroundcolor=\color{background},
    literate=
     *{0}{{{\color{numb}0}}}{1}
      {1}{{{\color{numb}1}}}{1}
      {2}{{{\color{numb}2}}}{1}
      {3}{{{\color{numb}3}}}{1}
      {4}{{{\color{numb}4}}}{1}
      {5}{{{\color{numb}5}}}{1}
      {6}{{{\color{numb}6}}}{1}
      {7}{{{\color{numb}7}}}{1}
      {8}{{{\color{numb}8}}}{1}
      {9}{{{\color{numb}9}}}{1}
      {:}{{{\color{punct}{:}}}}{1}
      {,}{{{\color{punct}{,}}}}{1}
      {\{}{{{\color{delim}{\{}}}}{1}
      {\}}{{{\color{delim}{\}}}}}{1}
      {[}{{{\color{delim}{[}}}}{1}
      {]}{{{\color{delim}{]}}}}{1},
}

% Configuración de los colores de los enlaces
\hypersetup{
    colorlinks = true,
    linkcolor = blue,
    filecolor = magenta,      
    urlcolor = cyan,
}

\begin{document}

\renewcommand\listtablename{Índice de Tablas}    
\renewcommand\listfigurename{Índice de Figuras}    

%%%%%%%%%%%%%%%%%%%%%%%%%%%%%%%%%%%%%%%%%%%%%%%%%%%%%%%%%%%%%%%%%%%%%%%%%%%%%%%
% First Page
%%%%%%%%%%%%%%%%%%%%%%%%%%%%%%%%%%%%%%%%%%%%%%%%%%%%%%%%%%%%%%%%%%%%%%%%%%%%%%%
\pagestyle{empty}
\thispagestyle{empty}


\newcommand{\HRule}{\rule{\linewidth}{1mm}}
\setlength{\parindent}{0mm}
\setlength{\parskip}{0mm}

\vspace*{\stretch{0.5}}

\begin{center}
\includegraphics[scale=0.8]{images/escuela-ingenieria-tecnologia-original}\\[10mm]
{\Huge Trabajo de Fin de Grado}
\end{center}

\HRule
\begin{flushright}
    {\Huge Extending GitHub CLI with Default Owners} \\[2.5mm]
    {\Large \textit Extendiendo GitHub CLI con Propietarios por Defecto} \\[5mm]
    {\Large Diego Pérez García} \\[5mm]
\end{flushright}
\HRule

\vspace*{\stretch{2}}
\begin{center}
  \Large La Laguna, \today
\end{center}

\setlength{\parindent}{5mm}

%%%%%%%%%%%%%%%%%%%%%%%%%%%%%%%%%%%%%%%%%%%%%%%%%%%%%%%%%%
% Signature page (add the official stamp)
%%%%%%%%%%%%%%%%%%%%%%%%%%%%%%%%%%%%%%%%%%%%%%%%%%%%%%%%%%
\newpage
\thispagestyle{empty}

D. {\bf Casiano Rodríguez León}, con N.I.F. 42.020.072-S  profesor Catedrático de Universidad adscrito al Departamento de Nombre del Departamento de la Universidad de La Laguna, como tutor


\bigskip
\bigskip
{\bf C E R T I F I C A N}

\bigskip
\bigskip
Que la presente memoria titulada:

\bigskip
''{\it Extending GitHub CLI with Default Owners}''

\bigskip
\bigskip
\bigskip

\noindent ha sido realizada bajo su dirección por D. {\bf Diego Pérez García},
con N.I.F. 79081733M.

\bigskip
\bigskip

Y para que así conste, en cumplimiento de la legislación vigente y a los efectos
oportunos firman la presente en La Laguna a \today

%%%%%%%%%%%%%%%%%%%%%%%%%%%%%%%%%%%%%%%%%%%%%%%%%%%%%%%%%%
\newpage
\thispagestyle{empty}

%{ \flushright

\begin{LARGE}
Agradecimientos
\end{LARGE}

\hspace{3mm}

\begin{large}
A mis amigos ya graduados que me han brindado apoyo, consejo y ánimos a pesar de haberme tardado un año más que ellos. \\
A mi mejor amiga por brindarme el mayor de los apoyos cuando tuve que repetir asignaturas. \\
A su vez a mis compañeros por esta gran experiencia y esos recuerdos que me llevo de mi vida universitaria. \\
A los profesores Francisco de Sande González, Marcos Colebrook-Santamaria, Iván Castilla Rodríguez y Albano José González Fernández por su gran labor como profesores demostrando cada año que tratan de llegar a los alumnos para transmitir sus conocimientos de la mejor manera. \\
A mi familia que me ha permitido centrarme en los estudios que me han llevado a ser el programador que soy hoy. \\
Y por último a mi tutor Casiano Rodríguez León por ofrecer esta oportunidad de proyecto y aprendizaje, a la vez de ser un profesor con paciencia y que me ha permitido sacar el mejor resultado. \\
\end{large}

%}
%%%%%%%%%%%%%%%%%%%%%%%%%%%%%%%%%%%%%%%%%%%%%%%%%%%%%%%%
\newpage
\thispagestyle{empty}

\bigskip
\begin{LARGE}
Licencia
\end{LARGE}

\begin{center}
\includegraphics[scale=1.8]{images/by_88x31}\\[5mm]
\end{center}

\begin{large}
© Esta obra está bajo una licencia de Creative Commons Reconocimiento 4.0 Internacional.
\end{large}

%%%%%%%%%%%%%%%%%%%%%%%%%%%%%%%%%%%%%%%%%%%%%%%%%%%%%%%%
\newpage 
\thispagestyle{empty}

\begin{abstract} % TODO: Revisar ortografía
Está muy arraigado en la informática el uso de GitHub como plataforma de enseñanza.
Es así que esta promueve las iniciativas de educación con GitHub Education. 
Dando paso a herramientas claves como GitHub Classroom con soporte para asignar tareas de manera fácil
y sencilla. Por otra parte la plataforma ofrece una herramienta llamada GitHub CLI que permite interactuar 
con esta por medio de  de la línea de comandos.

% XXXX Modificar mal expresado
Este trabajo presenta un comando nuevo para la GitHub CLI 
denominado \verb|gh owner| que tiene como objetivo romper con un problema 
de experiencia de usuario que es tener que recordar los largos y varios OWNERS 
que pueda tener un usuario en su cuenta. Esto resuelve entonces una cantidad de comandos 
que requieren de pasar como argumento OWNER/REPO, por lo que declarar un OWNER por defecto y simplemente pasar REPO resulta en combinar este OWNER con el REPO para generar un argumento completo y más fácil de rellenar.
\begin{palabrasClave}
github-cli, github-education, graphql, rest, git, go
\end{palabrasClave}

\end{abstract}
%%%%%%%%%%%%%%%%%%%%%%%%%%%%%%%%%%%%%%%%%%%%%%%%%%%%%%%%%
\newpage 
\vspace*{200px}
\thispagestyle{empty}

\begin{summary} % TODO: Traducir
\begin {keywords}
github-cli, github-education, graphql, rest, git, go
\end {keywords}

\end{summary}
%%%%%%%%%%%%%%%%%%%%%%%%%%%%%%%%%%%%%%%%%%%%%%%%%%%%%%%%%
\newpage{\pagestyle{empty}}
\thispagestyle{empty}

%%%%%%%%%%%%%%%%%%%%%%%%%%%%%%%%%%%%%%%%%%%%%%%%%%%%%%%%%
\pagestyle{myheadings} %my head defined by markboth or markright
% No funciona bien \markboth sin "twoside" en \documentclass, pero al
% ponerlo se dan un monton de errores de underfull \vbox, con lo que no se
% ha puesto.


%%Aqui debería poner el nombre del proyecto pero, como es muy grande no cabe y se ve feo en el PDF
\markboth{xxxxx}{}

%%%%%%%%%%%%%%%%%%%%%%%%%%%%%%%%%%%%%%%%%%%%%%%%%%%%%%%%%
%Numeracion en romanos
\renewcommand{\thepage}{\roman{page}}
\setcounter{page}{1}
\pagestyle{plain} 

%%%%%%%%%%%%%%%%%%%%%%%%%%%%%%%%%%%%%%%%%%%%%%%%%%%%%%%%%

{\hypersetup{linkcolor=black}
\tableofcontents

%%%%%%%%%%%%%%%%%%%%%%%%%%%%%%%%%%%%%%%%%%%%%%%%%%%%%%%%%
\newpage{\pagestyle{empty}}

\listoffigures

%%%%%%%%%%%%%%%%%%%%%%%%%%%%%%%%%%%%%%%%%%%%%%%%%%%%%%%%%
\newpage{\pagestyle{empty}}

\listoftables
}

%%%%%%%%%%%%%%%%%%%%%%%%%%%%%%%%%%%%%%%%%%%%%%%%%%%%%%%%%%%%%%%%%%%%%%%%%%%%%%%
\newpage{\pagestyle{empty}}

%%%%%%%%%%%%%%%%%%%%%%%%%%%%%%%%%%%%%%%%%%%%%%%%%%%%%%%%%%%%%%%%%%%%%%%%%%%%%%%
\newpage
\thispagestyle{empty}

%Numeracion a partir del capitulo I
\renewcommand{\thepage}{\arabic{page}}
\setcounter{page}{1}
\pagestyle{plain}

\chapter{\LARGE Introducción, Antecedentes y Estado del Arte}
\label{chapter:intro}

\section{GitHub CLI}
Desde hace varios años, GitHub ha sido una de las principales plataformas de repositorios en el ámbito de la programación. Esta plataforma incluye numerosas herramientas para el trabajo en equipo, el desarrollo de proyectos y el almacenamiento en la nube. GitHub siempre se ha esforzado por mejorar la experiencia del usuario, tratando de abarcar todos los estilos de trabajo de los programadores.

En este contexto, en febrero de 2020, se lanzó la GitHub CLI (Command Line Interface), también conocida como gh-cli o gh. Esta herramienta de código abierto permite interactuar con GitHub a través de la línea de comandos. Con GitHub CLI, es posible ver, crear, clonar y bifurcar repositorios, así como crear, cerrar, editar y ver incidencias, entre otras funciones. Sin embargo, el proyecto no se limita a las interacciones básicas con GitHub, ya que, aunque no incluye todas las funcionalidades de la interfaz web, también provee herramientas para generar extensiones que aumenten su potencial.

En este trabajo se presenta un nuevo comando para GitHub CLI denominado gh owner. Este comando tiene como objetivo solucionar un problema relacionado con la experiencia del usuario: la necesidad de recordar los numerosos y largos nombres de OWNERS asociados a una cuenta. El comando gh owner permite declarar un OWNER por defecto y especificar solo el REPO, combinando automáticamente el OWNER predeterminado con el REPO para generar un argumento completo. Esto facilita su uso y mejora la eficiencia en la gestión de repositorios.

\section{Antecedentes}
GitHub se ha consolidado como una herramienta esencial para el desarrollo de software colaborativo, proporcionando una plataforma robusta para la gestión de repositorios y la colaboración entre desarrolladores. Desde su creación en 2008, GitHub ha evolucionado para incluir una amplia gama de funcionalidades que facilitan la integración continua, la gestión de proyectos y el control de versiones, entre otros aspectos clave del desarrollo de software.

En febrero de 2020, GitHub lanzó la GitHub CLI (Command Line Interface) como una herramienta de código abierto para permitir a los desarrolladores interactuar con GitHub directamente desde la línea de comandos. La CLI fue diseñada para facilitar tareas comunes de GitHub sin necesidad de utilizar un navegador web, mejorando así la eficiencia y la productividad de los desarrolladores. Desde su lanzamiento, GitHub CLI ha evolucionado con actualizaciones periódicas que han añadido nuevas funcionalidades y mejoras basadas en los comentarios de la comunidad de usuarios.

\section{Estado del Arte}
El estado del arte en herramientas de línea de comandos para la gestión de repositorios y la interacción con plataformas de desarrollo ha avanzado significativamente en los últimos años. Las herramientas CLI se han vuelto indispensables para desarrolladores que buscan automatizar flujos de trabajo y mejorar la eficiencia en sus proyectos.

GitHub CLI: Desde su lanzamiento, GitHub CLI ha permitido a los usuarios realizar una variedad de operaciones, como clonar repositorios, gestionar issues y pull requests, y visualizar el estado de los repositorios. A pesar de estas capacidades, la necesidad de recordar y especificar los nombres de los propietarios (OWNERS) y los repositorios (REPOS) ha sido un punto de fricción para muchos usuarios, especialmente aquellos con múltiples repositorios y organizaciones.

Otras Herramientas CLI: Además de GitHub CLI, existen otras herramientas de línea de comandos que sirven a propósitos similares, como GitLab CLI y Bitbucket CLI. Estas herramientas también han abordado problemas relacionados con la gestión de repositorios y la automatización de tareas, pero cada una tiene su propio enfoque y conjunto de características. No obstante, el problema específico de la gestión de nombres de propietarios predeterminados no ha sido abordado de manera efectiva en muchas de estas herramientas.

Extensibilidad: Una de las características más poderosas de GitHub CLI es su capacidad para ser extendida a través de plugins y comandos personalizados. Esto ha permitido a los desarrolladores crear soluciones específicas para sus necesidades particulares, como el comando `gh owner`, que facilita la gestión de nombres de propietarios predeterminados, simplificando la sintaxis y mejorando la experiencia del usuario.

\begin{verbatim}
Cap\'{\i }tulo 3.
(./use.tex) [5]
Cap\'{\i }tulo 4.
(./design.tex (/usr/local/texlive/2020/texmf-dist/tex/latex/bera/t1fvm.fd)

! Package inputenc Error: Unicode character ́ (U+0301)
(inputenc)                not set up for use with LaTeX.

See the inputenc package documentation for explanation.
Type  H <return>  for immediate help.
 ...                              
\end{verbatim}

%%%%%%%%%%%%%%%%%%%%%%%%%%%%%%%%%%%%%%%%%%%%%%%%%%%%%%%%%%%%%%%%%%%%%%%%%%%%%%%
\newpage{\pagestyle{empty}}
\thispagestyle{empty}

\chapter{\LARGE Tecnologías}
\label{chapter:dos}

En el desarrollo de proyectos informáticos, existe una amplia gama de tecnologías disponibles que abarcan desde herramientas de control de versiones y plataformas de colaboración hasta lenguajes de programación y sistemas de preparación de documentos. La selección de las tecnologías adecuadas es crucial para asegurar la eficiencia, la escalabilidad y el éxito del proyecto. A continuación, se describen las principales tecnologías utilizadas en el desarrollo del comando gh owner para la GitHub CLI, proporcionando una visión general de sus características y su relevancia para el proyecto.

\section{GitHub}
GitHub es una plataforma de desarrollo colaborativo que utiliza Git para el control de versiones. Fundada en 2008, se ha convertido en una de las herramientas más populares para la gestión de proyectos de software, ofreciendo funcionalidades como la integración continua, el seguimiento de problemas, y la revisión de código. GitHub permite a los desarrolladores trabajar en conjunto de manera eficiente, manteniendo un historial completo de los cambios en el código.

\subsection{GitHub CLI} 
GitHub CLI, lanzada en febrero de 2020, es una herramienta de línea de comandos que permite a los usuarios interactuar con GitHub directamente desde su terminal. Esta herramienta facilita muchas operaciones comunes en GitHub, como clonar repositorios, gestionar issues y pull requests, y ver el estado de los repositorios. Al ofrecer una interfaz de línea de comandos, GitHub CLI permite a los desarrolladores integrar las funcionalidades de GitHub en sus scripts y flujos de trabajo automatizados, mejorando la eficiencia y la productividad.

Las capacidades de GitHub CLI pueden ser extendidas mediante la creación de comandos personalizados y plugins. Esto permite a los desarrolladores adaptar la herramienta a sus necesidades específicas, incorporando nuevas funcionalidades que mejoran la experiencia del usuario y resuelven problemas específicos, como el comando `gh owner` que simplifica la gestión de nombres de propietarios predeterminados.

\subsection{Las APIs de GitHub: REST y  GraphQL}
La API de GitHub proporciona un acceso programático a muchas de las funciones disponibles en la interfaz web de GitHub. A través de la API REST y GraphQL, los desarrolladores pueden realizar operaciones como la creación y gestión de repositorios, la administración de issues y pull requests, y la consulta de datos sobre usuarios y organizaciones.

La API de GitHub es una pieza fundamental en la extensión de GitHub CLI, permitiendo que los comandos personalizados interactúen con GitHub de manera efectiva. Al utilizar la API, los desarrolladores pueden automatizar tareas y construir herramientas que aprovechen la amplia funcionalidad de GitHub, mejorando así la eficiencia y la capacidad de gestión de sus proyectos.

\section{Lenguaje GO}
GO, también conocido como Golang, es un lenguaje de programación desarrollado por Google en 2007. Es conocido por su simplicidad, eficiencia y capacidad para manejar la concurrencia de manera efectiva. GO ha sido adoptado ampliamente en el desarrollo de software de sistemas y aplicaciones de red debido a su rendimiento y facilidad de uso.

El comando `gh owner` ha sido desarrollado utilizando GO, aprovechando sus características de concurrencia y su eficiencia en la ejecución. La decisión de utilizar GO se tomó debido a que la GitHub CLI está escrita en este lenguaje, lo que asegura una integración nativa y fluida con la herramienta existente.

\section{Overleaf y LaTeX}
Overleaf es una plataforma de escritura y colaboración en línea que utiliza LaTeX, un sistema de preparación de documentos de alta calidad. LaTeX es ampliamente utilizado en la elaboración de documentos académicos, técnicos y científicos debido a su capacidad para manejar la composición de textos complejos y la inclusión de fórmulas matemáticas.

En este trabajo, Overleaf y LaTeX se utilizan para la redacción y formateo de la documentación técnica. La combinación de estas herramientas permite una colaboración eficiente y una presentación profesional de los resultados del proyecto. LaTeX proporciona un control preciso sobre el diseño y la tipografía, asegurando que el documento final cumpla con los estándares académicos y profesionales requeridos.

Esta sección proporciona una descripción detallada de las tecnologías empleadas, resaltando sus características clave y su relevancia para el desarrollo del comando `gh owner` y la documentación del proyecto.

%%%%%%%%%%%%%%%%%%%%%%%%%%%%%%%%%%%%%%%%%%%%%%%%%%%%%%%%%%%%%%%%%%%%%%%%%%%%%%%
\newpage{\pagestyle{empty}}
\thispagestyle{empty}

\chapter{\LARGE Modo de uso}
\label{chapter:modo-de-uso}

En el siguiente capítulo se tratará como instalar y ejecutar el comando incluido `gh owner` en la GitHub CLI.Cabe destacar que `gh owner` no ha sido añadido a la GitHub CLI oficial, por lo que este capítulo es una simple demostración. Más adelante se explicará la implementación del comando y cómo ejecutarlo.

\section{Instalar gh}

\section{Ejecutar gh owner}

\section{Ejemplos de uso}


%%%%%%%%%%%%%%%%%%%%%%%%%%%%%%%%%%%%%%%%%%%%%%%%%%%%%%%%%%%%%%%%%%%%%%%%%%%%%%%
\newpage{\pagestyle{empty}}
\thispagestyle{empty}

\chapter{\LARGE Diseño}
\label{chapter:tres}

El diseño del comando gh owner se ha adaptado cuidadosamente al estilo de diseño de la GitHub CLI, manteniendo la coherencia con la estructura y usabilidad de la herramienta. Mientras que algunos comandos de gh incluyen subcomandos para organizar múltiples funcionalidades, en este caso se optó por implementar un único comando sin subcomandos. Esta decisión se tomó porque todas las funcionalidades requeridas se pueden resolver eficientemente mediante el uso de flags, simplificando así la sintaxis y mejorando la experiencia del usuario al reducir la complejidad operativa.

%%%%%%%%%%%%%%%%%%%%%%%%%%%%%%%%%%%%%%%%%%%%%%%%%%%%%%%%%
\newpage{\pagestyle{empty}}
\thispagestyle{empty}

\chapter{\LARGE Implementación}
\label{chapter:cuatro}


\section{Instalar GO}
El primer paso para utilizar el comando es asegurarse de que Go (Golang) está instalado en el sistema. Para ello, siga estos pasos:

1. Descargue la última versión de Go desde la página oficial: https://golang.org/dl/. \\
2. Siga las instrucciones de instalación correspondientes a su sistema operativo (Windows, macOS, Linux). \\
3. Verifique la instalación abriendo una terminal y ejecutando el siguiente comando:

\begin{verbatim}
  go version
\end{verbatim}

Si Go está instalado correctamente, verá una salida con la versión de Go instalada.

\section{Descargar el repositorio}

Dado que este proyecto no es una modificación directa de la GitHub CLI oficial, es necesario clonar el repositorio desde un fork y acceder a la rama de desarrollo. Para ello, siga los siguientes pasos:

1. Abra una terminal y clone el repositorio desde el fork con el siguiente comando:

\begin{verbatim}
  git clone git@github.com:gh-cli-for-education/cli.git
\end{verbatim}

2. Acceda al directorio del repositorio clonado:

\begin{verbatim}
  cd cli
\end{verbatim}

3. Cambie a la rama de desarrollo específica:

\begin{verbatim}
git checkout gh-owner-dev
\end{verbatim}

\subsection{Compilar el proyecto}

Una vez dentro del repositorio clonado, puede compilar el proyecto utilizando el comando `make`. Siga estos pasos:

1. Asegúrese de estar en el directorio raíz del repositorio clonado.
2. Ejecute el comando `make` para iniciar el proceso de compilación:

\begin{verbatim}
make
\end{verbatim}

Este comando se encargará de construir el proyecto y generar los binarios necesarios.

3. Después de la compilación exitosa, puede utilizar el binario generado para ejecutar el comando. En lugar de utilizar el comando `gh` estándar, utilice el binario ubicado en el directorio `bin` del proyecto:

\begin{verbatim}
./bin/gh <subcomando>
\end{verbatim}

Con estos pasos, el entorno estará configurado correctamente y podrá empezar a utilizar el comando desarrollado. En los siguientes capítulos se proporcionarán más detalles sobre la implementación y funcionalidades adicionales.

%%%%%%%%%%%%%%%%%%%%%%%%%%%%%%%%%%%%%%%%%%%%%%%%%%%%%%%%%
\newpage{\pagestyle{empty}}
\thispagestyle{empty}

\chapter{\LARGE Conclusiones y líneas futuras}
\label{chapter:Resultados}

Uno de los aspectos más destacados de este proyecto ha sido la preservación del estilo de trabajo característico de la GitHub CLI en el desarrollo del comando gh owner. A pesar de que inicialmente este no era el objetivo principal del proyecto, esta circunstancia ha representado una excelente oportunidad para adquirir nuevos conocimientos, mejorar habilidades y desarrollar competencias en un entorno diferente.

En cuanto a la funcionalidad actual del comando, existen varias ideas que podrían implementarse para aumentar su capacidad y hacerlo aún más completo. Sería bueno seguir aumentando esta funcionalidad y aprendiendo más dentro de este proyecto. Además de añadir un pr al repositorio para ver si sería aceptado.

%%%%%%%%%%%%%%%%%%%%%%%%%%%%%%%%%%%%%%%%%%%%%%%%%%%%%%%%%
\newpage{\pagestyle{empty}}
\thispagestyle{empty}

\chapter{\LARGE Summary and Conclusions}
\label{chapter:Conclusiones}

One of the most outstanding aspects of this project has been the preservation of the characteristic working style of GitHub CLI in the development of the `gh owner` command. Although initially this was not the main objective of the project, this situation has represented an excellent opportunity to acquire new knowledge, improve skills, and develop competencies in a different environment.

Regarding the current functionality of the command, there are several ideas that could be implemented to enhance its capability and make it even more comprehensive. It would be beneficial to continue expanding this functionality and gaining further insights within this project. Additionally, proposing a pull request to the repository to assess its potential acceptance would be advantageous.

%%%%%%%%%%%%%%%%%%%%%%%%%%%%%%%%%%%%%%%%%%%%%%%%%%%%%%%%%
\newpage{\pagestyle{empty}}
\thispagestyle{empty}

\chapter{\LARGE Presupuesto}
\label{chapter:presupuesto}

\input{budget.tex}

%%%%%%%%%%%%%%%%%%%%%%%%%%%%%%%%%%%%%%%%%%%%%%%%%%%%%%%%%
\newpage{\pagestyle{empty}\cleardoublepage}
\thispagestyle{empty}

\begin{appendix}

\chapter{\LARGE Repositorios del ecosistema}
\label{appendix:1}
Se adjuntan los enlaces a los repositorios de GitHub de los diferentes códigos en los que se ha trabajado.

\begin{itemize}
  \item XXXX
  El código de \verb|gh-edu| se encuentra en

  \href{https://github.com/gh-cli-for-education/gh-edu}{{\tt https://github.com/gh-cli-for-education/gh-edu}}

  
\end{itemize}

\end{appendix}
%\printnoidxglossaries
\printglossary[title=Glosario, toctitle=Glosario]

%%%%%%%%%%%%%%%%%%%%%%%%%%%%%%%%%%%%%%%%%%%%%%%%%%%%%%%%%%
%\begin{thebibliography}{X}
% Aquí figurará la bibliografía
%\end{thebibliography}
%\bibliographystyle{plain}
\bibliographystyle{unsrt}

\bibliography{memtfg} 
%\bibliography{bib.tex}
%%%%%%%%%%%%%%%%%%%%%%%%%%%%%%%%%%%%%%%%%%%%%%%%%%%%%%%%%%

\end{document}

