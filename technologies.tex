En el desarrollo de proyectos informáticos, existe una amplia gama de tecnologías disponibles que abarcan desde herramientas de control de versiones y plataformas de colaboración hasta lenguajes de programación y sistemas de preparación de documentos. La selección de las tecnologías adecuadas es crucial para asegurar la eficiencia, la escalabilidad y el éxito del proyecto. A continuación, se describen las principales tecnologías utilizadas en el desarrollo del comando gh owner para la GitHub CLI, proporcionando una visión general de sus características y su relevancia para el proyecto.

\section{GitHub}
GitHub es una plataforma de desarrollo colaborativo que utiliza Git para el control de versiones. Fundada en 2008, se ha convertido en una de las herramientas más populares para la gestión de proyectos de software, ofreciendo funcionalidades como la integración continua, el seguimiento de problemas, y la revisión de código. GitHub permite a los desarrolladores trabajar en conjunto de manera eficiente, manteniendo un historial completo de los cambios en el código.

\subsection{GitHub CLI} 
GitHub CLI, lanzada en febrero de 2020, es una herramienta de línea de comandos que permite a los usuarios interactuar con GitHub directamente desde su terminal. Esta herramienta facilita muchas operaciones comunes en GitHub, como clonar repositorios, gestionar issues y pull requests, y ver el estado de los repositorios. Al ofrecer una interfaz de línea de comandos, GitHub CLI permite a los desarrolladores integrar las funcionalidades de GitHub en sus scripts y flujos de trabajo automatizados, mejorando la eficiencia y la productividad.

Las capacidades de GitHub CLI pueden ser extendidas mediante la creación de comandos personalizados y plugins. Esto permite a los desarrolladores adaptar la herramienta a sus necesidades específicas, incorporando nuevas funcionalidades que mejoran la experiencia del usuario y resuelven problemas específicos, como el comando `gh owner` que simplifica la gestión de nombres de propietarios predeterminados.

\subsection{Las APIs de GitHub: REST y  GraphQL}
La API de GitHub proporciona un acceso programático a muchas de las funciones disponibles en la interfaz web de GitHub. A través de la API REST y GraphQL, los desarrolladores pueden realizar operaciones como la creación y gestión de repositorios, la administración de issues y pull requests, y la consulta de datos sobre usuarios y organizaciones.

La API de GitHub es una pieza fundamental en la extensión de GitHub CLI, permitiendo que los comandos personalizados interactúen con GitHub de manera efectiva. Al utilizar la API, los desarrolladores pueden automatizar tareas y construir herramientas que aprovechen la amplia funcionalidad de GitHub, mejorando así la eficiencia y la capacidad de gestión de sus proyectos.

\section{Lenguaje GO}
GO, también conocido como Golang, es un lenguaje de programación desarrollado por Google en 2007. Es conocido por su simplicidad, eficiencia y capacidad para manejar la concurrencia de manera efectiva. GO ha sido adoptado ampliamente en el desarrollo de software de sistemas y aplicaciones de red debido a su rendimiento y facilidad de uso.

El comando `gh owner` ha sido desarrollado utilizando GO, aprovechando sus características de concurrencia y su eficiencia en la ejecución. La decisión de utilizar GO se tomó debido a que la GitHub CLI está escrita en este lenguaje, lo que asegura una integración nativa y fluida con la herramienta existente.

\section{Overleaf y LaTeX}
Overleaf es una plataforma de escritura y colaboración en línea que utiliza LaTeX, un sistema de preparación de documentos de alta calidad. LaTeX es ampliamente utilizado en la elaboración de documentos académicos, técnicos y científicos debido a su capacidad para manejar la composición de textos complejos y la inclusión de fórmulas matemáticas.

En este trabajo, Overleaf y LaTeX se utilizan para la redacción y formateo de la documentación técnica. La combinación de estas herramientas permite una colaboración eficiente y una presentación profesional de los resultados del proyecto. LaTeX proporciona un control preciso sobre el diseño y la tipografía, asegurando que el documento final cumpla con los estándares académicos y profesionales requeridos.

Esta sección proporciona una descripción detallada de las tecnologías empleadas, resaltando sus características clave y su relevancia para el desarrollo del comando `gh owner` y la documentación del proyecto.