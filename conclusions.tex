``Software Engineering Is Programming Integrated Over Time''

\bigskip
What stands out the most in this final degree project has been the task of designing a modular ecosystem from scratch with some set goals and with the ideal of creating a software that does not need changes in its structure to grow, that is simple and intuitive, but useful and that allows collaboration. Achieving all these features
required an extensive design and prototyping stage, where ideas were implemented and discarded alike.

Also, another time-consuming section was error handling. Being an application that makes a lot of calls, reads and writes from files and asks for user input, there are many points where the application can fail. Handling these errors and polishing the application in general is work that does not generate an immediate result, but it does make the application more maintainable and stable.
On the other hand, I feel very fortunate to have been able to work with technologies that I feel comfortable with such as JavaScript, Go and fzf, but there have been more that I have learned and given the good experience I've had with: jq, gh cli and GraphQL, I'm sure I'll use them in future projects.

However, I am sorry I have not been able to meet the goal of interoperability between commands. The view command could benefit enormously if it could read from the standard input to receive information from a command such as data, but I preferred to focus my time on improving the quality and usability of the whole ecosystem.

The final degree work is finished, but the project can grow as much as I want. This type of project requires constant and prolonged work in order to be accepted by the community. My intentions are to continue contributing to this project and take it as my personal project and my way to contribute to open source.