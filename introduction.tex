\section{GitHub CLI}
Desde hace varios años, GitHub ha sido una de las principales plataformas de repositorios en el ámbito de la programación. Esta plataforma incluye numerosas herramientas para el trabajo en equipo, el desarrollo de proyectos y el almacenamiento en la nube. GitHub siempre se ha esforzado por mejorar la experiencia del usuario, tratando de abarcar todos los estilos de trabajo de los programadores.

En este contexto, en febrero de 2020, se lanzó la GitHub CLI (Command Line Interface), también conocida como gh-cli o gh. Esta herramienta de código abierto permite interactuar con GitHub a través de la línea de comandos. Con GitHub CLI, es posible ver, crear, clonar y bifurcar repositorios, así como crear, cerrar, editar y ver incidencias, entre otras funciones. Sin embargo, el proyecto no se limita a las interacciones básicas con GitHub, ya que, aunque no incluye todas las funcionalidades de la interfaz web, también provee herramientas para generar extensiones que aumenten su potencial.

En este trabajo se presenta un nuevo comando para GitHub CLI denominado gh owner. Este comando tiene como objetivo solucionar un problema relacionado con la experiencia del usuario: la necesidad de recordar los numerosos y largos nombres de OWNERS asociados a una cuenta. El comando gh owner permite declarar un OWNER por defecto y especificar solo el REPO, combinando automáticamente el OWNER predeterminado con el REPO para generar un argumento completo. Esto facilita su uso y mejora la eficiencia en la gestión de repositorios.

\section{Antecedentes}
GitHub se ha consolidado como una herramienta esencial para el desarrollo de software colaborativo, proporcionando una plataforma robusta para la gestión de repositorios y la colaboración entre desarrolladores. Desde su creación en 2008, GitHub ha evolucionado para incluir una amplia gama de funcionalidades que facilitan la integración continua, la gestión de proyectos y el control de versiones, entre otros aspectos clave del desarrollo de software.

En febrero de 2020, GitHub lanzó la GitHub CLI (Command Line Interface) como una herramienta de código abierto para permitir a los desarrolladores interactuar con GitHub directamente desde la línea de comandos. La CLI fue diseñada para facilitar tareas comunes de GitHub sin necesidad de utilizar un navegador web, mejorando así la eficiencia y la productividad de los desarrolladores. Desde su lanzamiento, GitHub CLI ha evolucionado con actualizaciones periódicas que han añadido nuevas funcionalidades y mejoras basadas en los comentarios de la comunidad de usuarios.

\section{Estado del Arte}
El estado del arte en herramientas de línea de comandos para la gestión de repositorios y la interacción con plataformas de desarrollo ha avanzado significativamente en los últimos años. Las herramientas CLI se han vuelto indispensables para desarrolladores que buscan automatizar flujos de trabajo y mejorar la eficiencia en sus proyectos.

GitHub CLI: Desde su lanzamiento, GitHub CLI ha permitido a los usuarios realizar una variedad de operaciones, como clonar repositorios, gestionar issues y pull requests, y visualizar el estado de los repositorios. A pesar de estas capacidades, la necesidad de recordar y especificar los nombres de los propietarios (OWNERS) y los repositorios (REPOS) ha sido un punto de fricción para muchos usuarios, especialmente aquellos con múltiples repositorios y organizaciones.

Otras Herramientas CLI: Además de GitHub CLI, existen otras herramientas de línea de comandos que sirven a propósitos similares, como GitLab CLI y Bitbucket CLI. Estas herramientas también han abordado problemas relacionados con la gestión de repositorios y la automatización de tareas, pero cada una tiene su propio enfoque y conjunto de características. No obstante, el problema específico de la gestión de nombres de propietarios predeterminados no ha sido abordado de manera efectiva en muchas de estas herramientas.

Extensibilidad: Una de las características más poderosas de GitHub CLI es su capacidad para ser extendida a través de plugins y comandos personalizados. Esto ha permitido a los desarrolladores crear soluciones específicas para sus necesidades particulares, como el comando `gh owner`, que facilita la gestión de nombres de propietarios predeterminados, simplificando la sintaxis y mejorando la experiencia del usuario.

\begin{verbatim}
Cap\'{\i }tulo 3.
(./use.tex) [5]
Cap\'{\i }tulo 4.
(./design.tex (/usr/local/texlive/2020/texmf-dist/tex/latex/bera/t1fvm.fd)

! Package inputenc Error: Unicode character ́ (U+0301)
(inputenc)                not set up for use with LaTeX.

See the inputenc package documentation for explanation.
Type  H <return>  for immediate help.
 ...                              
\end{verbatim}